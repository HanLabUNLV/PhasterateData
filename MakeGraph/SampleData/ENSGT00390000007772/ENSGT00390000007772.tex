% Creates a printout of the given alignment file in .fa format using the texshade package.
% This creates a "pretty" alignment graph. Additionally, this script expects a score file
% of phyloP scores to plot against every column in our alignment. This file should contain
% normalize P-Scores from -50.0 to 50.0 with a score as a floating point number at every
% line with the same number of scores as there are lines in the file.

% This script is used to plot conservation scores for gap envents visually alongside a DNA
% multiple sequence alignment, therefore the fasta file should have an "extra" specie as
% it's first specie containing no gaps. This specie will be used as the coordinate frame
% to make sure that the column numbers are correct.

% See attached files for example data.

\documentclass{minimal}
\usepackage{texshade}


%SPECIFY ALIGNMENT LENGTH HERE!!!
\newcommand{\alignmentLength}{1..831}
% Sample:
%         \newcommand{\alignmentLength}{1..831}


\begin{document}
%Given name of alignment file here.
\begin{texshade}{ENSGT00390000007772.fa}

  %Length of our alignment.
  \setends[1]{1}{\alignmentLength}
  \shadingmode[rasmol]{functional}

  %Original Colors are ugly, these are less constrasty.
  \funcshadingstyle{C}{Purple}{White}{upper}{up}
  \funcshadingstyle{G}{Cyan}{White}{upper}{up}
  \funcshadingstyle{T}{Orange}{White}{upper}{up}

  %Ignore the specie  as it is all N's. We  use this to have a global 
  %coordinate frame because we don't want to skip any species.
  \hideseq{1}

  \hideconsensus{}
  \gapchar{-}
  \hidenumbering{}
  \showruler{top}{1}
  \shadeallresidues{}

  %Add graph of scores based on file "normalized.txt", this file is expected
  %to have the same number of lines as there are sites in the alignment. The
  %values are expected to be normalized from -50.0 to 50.0.
  \feature{bbottom}{0}{\alignmentLength}{bar[-50.0,50.0]:normalized.indel.txt[Blue]}{}
  \showfeaturename{bbottom}{Indel}

  %Extra features may be added by using extra files with different positions.
  \feature{bottom}{0}{1..831}{bar[-50.0,50.0]:normalized.subst.txt[Blue]}{}
  \showfeaturename{bottom}{Subst}

  \featurenamescolor{Blue}

  \bargraphstretch{3}
\end{texshade}

\end{document}